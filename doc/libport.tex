%% Copyright (C) 2009-2010, Gostai S.A.S.
%%
%% This software is provided "as is" without warranty of any kind,
%% either expressed or implied, including but not limited to the
%% implied warranties of fitness for a particular purpose.
%%
%% See the LICENSE file for more information.

\documentclass[openright,twoside,11pt]{book}
  \usepackage{gostai-documentation}

\newenvironment{workaround}{\paragraph{Workaround}}{}

\title{Gostai Libport}
\subtitle{Version \VcsDescription}
\author{Gostai}

\begin{document}
\maketitle

\chapter{Libport}
\section{Compiler}

\subsection{ATTRIBUTE\_DLLEXPORT, ATTRIBUTE\_DLLIMPORT}

Use these guys to declare what symbols should be exported from the
shared-object you are building, and what symbols should be imported from the
shared-object you are using.  Of course, exported symbols at build-time are
imported symbols at use-time, so the headers should use a single symbol to
flag exported/imported symbols.  Use a CPP symbol to specify whether you are
building.  See for instance \file{libport/export.hh}:

\lstinputlisting[language=C++,basicstyle=\ttfamily\footnotesize]{include/libport/export.hh}

Things to know:
\begin{itemize}
\item Put it before the declaration of the functions you export:
\begin{cxx}
  LIBPORT_API backtrace_type& backtrace(backtrace_type& bt);
\end{cxx}

\item Put it between \lstinline|class| or \lstinline|struct| and the name of
  the class.
\begin{cxx}
  /// Defines the operator() for the classes get_type, set_type and swap_type.
  struct LIBPORT_API iomanipulator
    : public std::unary_function<void, std::ostream>
  {
    // ...
\end{cxx}

\item Also do it for inner classes (required by Visual):
\begin{cxx}
  template <class StoredType>
  class LIBPORT_API xalloc
  {
    class LIBPORT_API set_type : public iomanipulator
    {
    public:
    // ...
\end{cxx}
\end{itemize}

\chapter{Documentation}

\section{TeX4ht}

TeX4ht works well, but there is a number of limitations you should be
aware of if you want to stay out of troubles.
\begin{itemize}
\item The most detailed introductory documentation I found is
  \url{http://www.gutenberg.eu.org/pub/GUTenberg/publicationsPDF/37-popineau.pdf}.
\item The TeX4ht site does contain a complete documentation, it is
  just hard to realize where the content is.  See
  \url{http://www.cse.ohio-state.edu/~gurari/TeX4ht/mn11.html#QQ1-11-51}
  and note that the numbers on the left-hand side are hyperlinks to
  detailed information.
\item The author passed away.  The project is now hosted here:
  \url{https://puszcza.gnu.org.ua/projects/tex4ht}.
\end{itemize}

\subsection{Known Issues}

\begin{itemize}
\item Avoid starred sections at sectioning boundaries.  I.e., if you
  split at sections, you should avoid playing with
  \lstinline|\section*|.

\item Don't create environments with embed hlines in the close part
  that do not have their \lstinline|\\| before, it does not compile.
  It was reported.

  So this is wrong:
\begin{lstlisting}[language={[LaTeX]TeX}]
\newenvironment{operatorTable}
{%
  \begin{tabular}{|c|c|c|c|c|}%
  }{%
    \hline%
  \end{tabular}%
}

\begin{operatorTable}
  a & b & c & d & e\\
  a & b & c & d & e\\
\end{operatorTable}
\end{lstlisting}

  But this works:
\begin{lstlisting}[language={[LaTeX]TeX}]
\newenvironment{operatorTable}
{%
  \begin{tabular}{|c|c|c|c|c|}%
    \hline%
  }{%
    \\\hline%
  \end{tabular}%
}

\begin{operatorTable}
  a & b & c & d & e\\
  a & b & c & d & e
\end{operatorTable}
\end{lstlisting}
\end{itemize}

\chapter{Portability issues}

\section{Mac OS X and Linux}

It seems that in interactive sessions, the flags of stdin, stdout and stderr
are bound.  If you fcntl one of them, the others will be affected.  See the
following program.

\lstinputlisting[language=C++,basicstyle=\ttfamily\footnotesize]{fcntl.cc}

\paragraph{Default}
In an interactive session.  It's the same under Screen.
\begin{lstlisting}
$ ./a.out a
"----------------------": ----------------------
"Save": Save
fcntl(STDIN_FILENO, F_GETFL): 2
(flags = fcntl(STDOUT_FILENO, F_GETFL)): 2
fcntl(STDERR_FILENO, F_GETFL): 2
"Set": Set
(res = fcntl(STDOUT_FILENO, F_SETFL, res)): 0
fcntl(STDIN_FILENO, F_GETFL): 6
fcntl(STDOUT_FILENO, F_GETFL): 6
fcntl(STDERR_FILENO, F_GETFL): 6
a
"Restore": Restore
(res = fcntl(STDOUT_FILENO, F_SETFL, flags)): 0
fcntl(STDIN_FILENO, F_GETFL): 2
fcntl(STDOUT_FILENO, F_GETFL): 2
fcntl(STDERR_FILENO, F_GETFL): 2
\end{lstlisting}

\paragraph{Redirect stdout}
This time, stdin and stdout are not correlated.  The same happens if you
redirect stderr instead: it is detached from the two others.

\begin{lstlisting}
$ ./a.out a | cat
"----------------------": ----------------------
"Save": Save
fcntl(STDIN_FILENO, F_GETFL): 2
(flags = fcntl(STDOUT_FILENO, F_GETFL)): 1
fcntl(STDERR_FILENO, F_GETFL): 2
"Set": Set
(res = fcntl(STDOUT_FILENO, F_SETFL, res)): 0
fcntl(STDIN_FILENO, F_GETFL): 2
fcntl(STDOUT_FILENO, F_GETFL): 5
fcntl(STDERR_FILENO, F_GETFL): 2
"Restore": Restore
(res = fcntl(STDOUT_FILENO, F_SETFL, flags)): 0
fcntl(STDIN_FILENO, F_GETFL): 2
fcntl(STDOUT_FILENO, F_GETFL): 1
fcntl(STDERR_FILENO, F_GETFL): 2
a
\end{lstlisting}

\begin{lstlisting}
$ ./a.out a >/dev/null
"----------------------": ----------------------
"Save": Save
fcntl(STDIN_FILENO, F_GETFL): 2
(flags = fcntl(STDOUT_FILENO, F_GETFL)): 1
fcntl(STDERR_FILENO, F_GETFL): 2
"Set": Set
(res = fcntl(STDOUT_FILENO, F_SETFL, res)): 0
fcntl(STDIN_FILENO, F_GETFL): 2
fcntl(STDOUT_FILENO, F_GETFL): 5
fcntl(STDERR_FILENO, F_GETFL): 2
"Restore": Restore
(res = fcntl(STDOUT_FILENO, F_SETFL, flags)): 0
fcntl(STDIN_FILENO, F_GETFL): 2
fcntl(STDOUT_FILENO, F_GETFL): 1
fcntl(STDERR_FILENO, F_GETFL): 2
\end{lstlisting}

\section{Windows}

\subsection{Test suite}

The Libport test suite uses Boost.UnitTest.  While our builds (that
aim for a static linking of the Boost libraries) rely on the absence
of the Boost dlls in the library directory, this breaks the UnitTest
framework.

So we need \file{boost\_unit\_test\_framework-vc*.*} files, not just
\file{libboost\_unit\_test\_framework-vc.lib}.

\subsection{Killing processes}
\label{sec:win:kill}
Beware that the \command{kill} builtin from shells cannot kill Windows
processes, use \command{/bin/kill}.  Use the following shell function
for \command{killall}.

\begin{shell}
killall ()
{
  ps -W |
    perl -ne '
    BEGIN { $re = join "|", qw ('"$*"'); };
    /\s*(\d+).*(?:$re)/ && push @res, $1;
    END { print "@res\n"; print STDERR "$re: @res\n"; };' |
    xargs /bin/kill -f
}
\end{shell}

\subsection{Crashes}

Windows features an ``automatic bug reporting'' system that pops-up a
window and expects user interaction.  Many such windows can stack up
and consume resources (and the guilty process is still present).

To prevent such windows:

panneau de config; System; Advanced; Error Report; desactivate +
disable "But Notify...".

The process that displays the window is \command{dwwin.exe}.  It can
be killed (\autoref{sec:win:kill}).

\section{MSVC}

\subsection{error C2883}
MSVC 2005 dies on the following piece of code:

\begin{cxx}
#include <iostream>
#define ECHO(S) std::cerr << #S << std::endl

namespace libport
{
  struct Top
  {
    virtual ~Top() {}
    virtual void foo(int)    { ECHO(Top::foo(int)); }
    virtual void foo(double) { ECHO(Top::foo(double)); }
  };
}

int
main ()
{
  struct Bot : libport::Top
  {
    using Top::foo;
    virtual ~Bot() {}
    virtual void foo(double) { ECHO(Bot::foo(double)); }
  };

  Bot b;
  b.foo(12);
  b.foo(1.2);
}
\end{cxx}

\begin{shell}
$ cl.exe using.cc -o using
using.cc:21: error C2883: 'main::Bot::foo' :
                           function declaration conflicts with
                           'libport::Top::foo'
                           introduced by using-declaration
using.cc:10: note:         see declaration of 'libport::Top::foo'
\end{shell}

\begin{workaround}
  Extract the struct from the function.
\end{workaround}
\end{document}

%%% Local Variables:
%%% mode: latex
%%% TeX-master: t
%%% ispell-dictionary: "american"
%%% fill-column: 76
%%% End:
